\chapter{Análise da Potência CA}

\section{Potência instantânea e potência média}

A \textbf{potência instantânea}, $p(t)$, em watts, é a taxa na qual um elemento absorve energia. É a potência a qualquer instante.

\begin{equation}
 p(t) = v(t) \cdot i(t)
\end{equation}

Considerando $v(t) = V_m cos(\omega t + \theta_v)$ e $i(t) = I_m cos(\omega t + \theta_i)$ e aplicando uma identidade trigonométrica \footnote{$cos(A)cos(B) = \frac{1}{2}cos(A-B) + \frac{1}{2}cos(A+B)$}, podemos expressar a potência como:

\begin{equation} \label{potinst}
 p(t) = \frac{1}{2}V_m I_m \cos(\theta_v - \theta_i) + \cos(2 \omega t + \theta_v + \theta_i)
\end{equation}

\begin{tikzpicture}
\begin{axis}[axis lines = left, xlabel = t (s), ylabel = V]
\addplot[domain=0:360,samples=100,color=red]{sin(x)};
\end{axis}
\end{tikzpicture}

A \textbf{potência média}, em watts, é a média da potência instantânea ao longo de um período.

\begin{equation}
 P = \frac{1}{2} \int_0^T p(t) dt
\end{equation}

Para a expressão na Equação \ref{potinst}, temos que:

\begin{equation}
 P = \frac{1}{2} V_m I_m cos(\theta_v - \theta_i)
\end{equation}

<demonstração>

Note que $p(t)$ cvaria com o tempo, ao passo que $P$ não.

As formas fasoriais de $v(t)$ e $i(t)$  são, respectivamente, $\mathbf{V} = V_m \angle \theta_v$ e $\mathbf{I} = I_m \angle \theta_i$. Para usar fasores, percebemos que:

\begin{align}
 \frac{1}{2} \mathbf{V}\mathbf{I^*} &= \frac{1}{2} = V_m I_m \angle \theta_v - \theta_i \\
 &= \frac{1}{2} V_m I_m [\cos(\theta_v - \theta_i) + j \sin(\theta_v - \theta_i)]
\end{align}

Reconhecemos a parte real dessa expressão como a potência média, P.

Em um circuito resistivo, $\theta_v = \theta_i$, então, $\cos(\theta_v - \theta_i) = 1$, nos dando

\begin{equation}
 P = \frac{1}{2} V_m I_m = \frac{1}{2} {I_m}^2 R
\end{equation}

Quando $\theta_v - \theta_i = +- 90^{\circ}$, temos que $\cos(\theta_v - \theta_i) = 0$. Então, em um circuito putamente reativo, a potência média é zero.

<3 exemplos>

\section{Transferência de Potência média Máxima}

Considere o equivalente de Thévenin de um circuito:

<figura>

Na morma retangular, as impedâncias são:

\begin{align}
 Z_{Th} &= R_{Th} + j X_{Th} \\
 Z_L &= R_L + j X_L
\end{align}

A corrente através da carga será:

\begin{equation}
 \mathbf{I} = \frac{\mathbf{V}_{Th}}{Z_{Th} + Z_L} = \frac{\mathbf{V}_{Th}}{(R_{Th} + R_L) + j (X_{Th} + X_L)}
\end{equation}

A potência média na carga então será

\begin{equation} \label{potencia}
 P = \frac{1}{2} |\mathbf{I}|^2 \cdot R_L = \frac{|\mathbf{V}_{Th}| R_L / 2}{(R_{Th} + R_L)^2 + (X_{Th} + X_L)^2}
\end{equation}

Tomando $\partial P / \partial R_L$ e $\partial P / \partial X_L$ e igualando ambas expressões a zero, vamos ter:

%\begin{align}
% X_L &= - X_{Th} \label \\
% R_L &= \sqrt{{R_{Th}}^2 + (X_{Th} + X_L)^2}
%\end{align}

Combinando as Equações, conclui-se que para haver a máxima transferência,

\begin{equation}
 Z_L = {Z_{Th}}^{*}
\end{equation}

Fazendo $Z_L = {Z_{Th}}^{*}$ na Equação \eqref{potencia}, vamos obter a potência média máxima na carga:

\begin{equation}
 P_{\text{máx}} = \frac{|\mathbf{V}_{Th}|^2}{8 R_{Th}} 
\end{equation}

<exemplos>

\section{Valor Eficaz ou RMS}

\textbf{Valor eficaz} de uma corrente periódica é a corrente
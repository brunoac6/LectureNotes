\chapter{Limites e Continuidade}

Chama-se limite o comportamento de uma função $f(x)$ em torno de um valor $x$. Escrevemos
$$lim_{x \rightarrow a} f(x) = L$$
e dizemos "o limite de $f(x)$, quando $x$ tende a $a$ é igual a $L$" se pudermos tornar os valors de $f(x)$ arbitrariamente próximos de $L$ (tão próximos quanto quisermos), tornando $x$ suficientemente próximo de $a$ (por ambos os lados de $a$), mas não igual a $a$.

\begin{example}
Vamos investigar o comportamento de uma função $f$ definida por $f(x) = x^2 - x + 2$, para valores de $x$ próximos de 2.

\begin{tikzpicture}[scale=0.5]
\begin{axis}[
    axis lines = left,
    xlabel = $x$,
    ylabel = {$f(x)$},
]
%Below the red parabola is defined
\addplot [
    domain=0:3, 
    samples=100, 
    color=green,
]
{x^2 - x + 2};
\end{axis}
\end{tikzpicture}

\begin{table}
\begin{tabular}{|c|c||c|c|}
\hline
$x$ & $f(x)$ & $x$ & $f(x)$ \\
\hline 
1 & 2,000000 & 3 & 8,000000 \\
\hline
1,5 & 2,750000 & 2,5 & 5,750000 \\
\hline
1,9 & 3,710000 & 2,1 & 4,310000 \\
\hline
1,95 & 3,852500 & 2,05 & 4,152500 \\
\hline
1,99 & 3,970100 & 2,01 & 4,030099 \\
\hline
1,999 & 3,997001 & 2,001 & 4,003000 \\
\hline
\end{tabular}
\end{table}

\end{example}

